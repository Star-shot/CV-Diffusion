%!TEX program=xelatex

% 碰到Windows版本提示Fandol字体,可以在命令行中以管理员权限执行:tlmgr update -self -all
\documentclass[final]{cvpr}

\usepackage[UTF8]{ctex}

%\usepackage{cvpr}
\usepackage{times}
\usepackage{epsfig}
\usepackage{graphicx}
\usepackage{amsmath}
\usepackage{amssymb}
\usepackage{subfigure}
\usepackage{overpic}
\usepackage{booktabs} % 推荐添加:用于制作更专业的三线表

\usepackage{enumitem}
\setenumerate[1]{itemsep=0pt,partopsep=0pt,parsep=\parskip,topsep=5pt}
\setitemize[1]{itemsep=0pt,partopsep=0pt,parsep=\parskip,topsep=5pt}
\setdescription{itemsep=0pt,partopsep=0pt,parsep=\parskip,topsep=5pt}

\usepackage[pagebackref=true,breaklinks=true,colorlinks,bookmarks=false]{hyperref}

%\cvprfinalcopy % *** Uncomment this line for the final submission

\def\cvprPaperID{****} % *** 在此处输入论文ID
\def\confYear{CVPR 202X}
\def\httilde{\mbox{\tt\raisebox{-.5ex}{\symbol{126}}}}

% 自定义命令
\newcommand{\mypara}[1]{\paragraph{#1.}}
\renewcommand{\figref}[1]{图\ref{#1}}
\renewcommand{\tabref}[1]{表\ref{#1}}
\renewcommand{\equref}[1]{式\ref{#1}}
\renewcommand{\secref}[1]{第\ref{#1}节}

% 页面设置
\setcounter{page}{1}

\begin{document}

\def\abstract{\centerline{\large\bf 摘要} \vspace*{12pt} \it}

%%%%%%%%% 标题部分
\title{在此处输入论文标题}

\author{作者姓名$^{1}$\quad 作者姓名$^{1}$ \quad 作者姓名$^{2}$\\
    $^{1}$ 第一单位名称 \quad \quad
    $^2$ 第二单位名称\\
}

\maketitle

%%%%%%%%% 摘要
\begin{abstract}
在此处撰写摘要。摘要应简明扼要地概括论文的研究背景、提出的方法、主要贡献以及实验结果。
例如:我们提出了一种新的[方法名称],用于解决[具体问题]。该方法利用了[核心技术]。
实验结果表明,该方法在[数据集名称]上取得了优异的性能,优于现有的最先进方法。
\end{abstract}

%%%%%%%%% 正文部分

\section{引言}\label{sec:Introduction}
在此处撰写引言。介绍研究领域的背景、目前存在的问题、本文的动机以及主要贡献。

主要贡献总结如下:
\begin{itemize}
    \item 提出了......
    \item 改进了......
    \item 验证了......
\end{itemize}

\section{相关工作}\label{sec:RelatedWorks}
在此处综述相关工作。
可以将相关工作分类讨论,例如:
\mypara{基于A的方法} 描述此类方法的优缺点...
\mypara{基于B的方法} 描述此类方法的优缺点...

\section{本文方法}\label{sec:Method}
在此处详细描述你的算法或模型。

\subsection{问题定义}
使用数学公式定义问题。例如,给定输入图像 $I$,我们的目标是计算......
\begin{equation}\label{equ:Example}
    y = f(x) + \epsilon
\end{equation}

\subsection{算法细节}
详细描述算法步骤。

%%%%%%%%%%%%%%%%%%%%%%%%%%%%%%%%%%%%%%%%%%%%%%%%%%%%%%%%%%%%%%%%%%%%%%
% 实验部分(已替换为占位符)
%%%%%%%%%%%%%%%%%%%%%%%%%%%%%%%%%%%%%%%%%%%%%%%%%%%%%%%%%%%%%%%%%%%%%%
\section{实验结果与分析}\label{sec:Experiment}

我们在本节中展示实验设置、数据集介绍以及定性和定量的对比结果。

\subsection{实验设置}
\mypara{数据集} 我们在以下数据集上进行了测试:Dataset A, Dataset B...
\mypara{参数设置} 实验中使用的参数如下:$\alpha = 0.5, \beta = 0.9$...
\mypara{评价指标} 我们使用准确率(Precision)、召回率(Recall)以及F-measure来评价性能。

\subsection{对比实验}
我们将本文提出的方法与以下几种最先进的方法进行了对比:
Method A \cite{98pami/Itti}, Method B \cite{03ACMMM/Ma_Contrast-based} 等。(请确保在bib文件中添加对应的引用)

\subsubsection{运行时间对比}

% 表格占位符
\begin{table}[t]
    \centering
    \caption{不同方法在测试集上的平均运行时间对比(单位:秒)。我们的方法在保持精度的同时具有较快的速度。}
    \label{tab:TimeComparison}
    \begin{tabular}{l|c|c|c|c}
        \toprule
        方法 & Method A & Method B & Method C & \textbf{Ours} \\
        \midrule
        时间(s) & 0.XX & 1.XX & 0.XX & \textbf{0.XX} \\
        代码环境 & Matlab & C++ & Python & \textbf{Python} \\
        \bottomrule
    \end{tabular}
\end{table}

如\tabref{tab:TimeComparison}所示,我们的方法运行时间为...

\subsubsection{定量评估结果}

% 复杂表格占位符
\begin{table*}[t]
    \centering
    \caption{在主要数据集上的性能对比(占位数据)。加粗数字表示最优结果。}
    \label{tab:Performance}
    \begin{tabular}{l|ccc|ccc}
        \toprule
        & \multicolumn{3}{c|}{Dataset 1} & \multicolumn{3}{c}{Dataset 2} \\
        方法 & Precision & Recall & F-measure & Precision & Recall & F-measure \\
        \midrule
        Comparison 1 & 0.85 & 0.80 & 0.82 & 0.75 & 0.70 & 0.72 \\
        Comparison 2 & 0.88 & 0.82 & 0.85 & 0.78 & 0.75 & 0.76 \\
        Comparison 3 & 0.89 & 0.85 & 0.87 & 0.80 & 0.78 & 0.79 \\
        \midrule
        \textbf{Ours} & \textbf{0.92} & \textbf{0.88} & \textbf{0.90} & \textbf{0.85} & \textbf{0.82} & \textbf{0.83} \\
        \bottomrule
    \end{tabular}
\end{table*}

定量实验结果如\tabref{tab:Performance}所示。可以看出,我们的方法在各项指标上均优于对比方法...

\subsection{定性结果展示}

% 图片占位符
\begin{figure*}[t]
    \centering
    % 使用 demo 模式或者替换为你自己的文件名
    % \includegraphics[width=\textwidth]{results_comparison.pdf}
    \fbox{\parbox[c][4cm]{\textwidth}{\centering 图示占位符:此处放置多方法视觉对比图 \\ (例如:原图 | 真值 | 方法A | 方法B | 本文方法)}}
    \caption{不同方法在具有挑战性的场景下的视觉对比结果。(a) 输入图像, (b) 真值 (Ground Truth), (c) Method A, (d) Method B, (e) \textbf{本文方法}。可以看出我们的方法在边缘细节处理上更具优势。}
    \label{fig:VisualComparison}
\end{figure*}

\figref{fig:VisualComparison} 展示了部分可视化结果。在光照不均匀或背景复杂的情况下(如图中第一行所示),对比方法出现了误检,而我们的方法能够准确地...

\subsection{消融实验 (Ablation Study)}
为了验证模型中各模块的有效性,我们进行了消融实验。

\begin{figure}[t]
    \centering
    % \includegraphics[width=\columnwidth]{ablation_plot.pdf}
    \fbox{\parbox[c][5cm]{\columnwidth}{\centering 图示占位符:此处放置曲线图或柱状图 \\ (例如:PR曲线对比)}}
    \caption{消融实验的P-R曲线对比。Ours-Full表示完整模型,Ours-Base表示去掉核心模块后的基准模型。}
    \label{fig:Ablation}
\end{figure}

如图\ref{fig:Ablation}所示,移除[模块名称]后,性能下降了约X\%,这证明了该模块的重要性。

\section{总结与展望}\label{sec:Conclusion}
本文提出了一种......的方法。实验结果表明......
在未来的工作中,我们将致力于......

\paragraph{致谢.} 本项目受到了......基金的支持。

{\small
\bibliographystyle{ieee}
% 请确保当前目录下有名为 Saliency.bib 或你自己的 .bib 文件
\bibliography{Saliency}
}

\end{document}